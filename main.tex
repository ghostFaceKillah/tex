\documentclass[reprint]{revtex4-1}
\usepackage{graphicx}% Include figure files
\usepackage{polski}
\usepackage{array}
\usepackage[utf8]{inputenc}
\usepackage{bm}% bold math
\usepackage{float}
\usepackage{amssymb}
\usepackage{amsmath}
\usepackage[a4paper, margin = 0.75in, top=1.2in ]{geometry}
\newcommand{\spec}[2][c]{%
    \begin{tabular}[#1]{@{}c@{}}#2\end{tabular}}

\begin{document}

\title{ Nieklasyczne metody optymalizacji \\
Praca domowa nr 3}

\author{Michał Garmulewicz, 48024 }

\begin{abstract}
  Raport podsumowuje sugerowane alokacje towarów w magazynie klienta, które minimalizują drogę
  produktu wewnątrz obiektu klienta.

\end{abstract}

\maketitle

\section{Omówienie problemu decyzyjnego}
Właścicielem problemu decyzyjnego jest firma Northwest Newsprint, część koncernu Tabela~\ref{pic1}


\section{Lokalizacje}
\subsection{Zakłady produkcyjne}
Firma 

\begin{table}[H]
  \caption{Zakłady produkcyjne}
  \centering
  \begin{tabular}{rrrr}
      \hline
      Zakład & \spec{ Produkcja \\ dzienna (ton) } & \spec{ Produkcja \\ roczna (ton)}
              & \spec{ Koszt \\ produkcji tony } \\
        \hline
        Spruce Mills & 462 & 166320 & 390 \$ \\
        \hline
         \end{tabular}
       \end{table}


\begin{figure}[H]\label{tab_wyniki}
  \centering
  \includegraphics[width=1\linewidth]{doge.jpg}
  \caption{Porównanie luk produkcji}\label{pic1}
\end{figure}



\end{document}
